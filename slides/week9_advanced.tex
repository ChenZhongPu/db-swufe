\documentclass[aspectratio=169, 14pt]{beamer}
\usepackage[utf8]{inputenc}
\usepackage{xeCJK}
\usepackage{graphicx}
\usepackage{transparent}
\usepackage[ruled, lined, linesnumbered, commentsnumbered]{algorithm2e}
\usepackage{pgfplots}
\usepackage{tikz}
\usetikzlibrary{matrix,backgrounds}
\usetikzlibrary{arrows}
\usetikzlibrary {arrows.meta}
\usetikzlibrary{calc,shadows.blur,fit,positioning}
\usepackage{minted}
\usepackage{fontawesome5}
\usepackage{booktabs}
\usepackage{caption}
\usepackage{hyperref}
\hypersetup{
    colorlinks=true,
    linkcolor=blue,
    filecolor=magenta,      
    urlcolor=cyan,
    }
\urlstyle{same}
\usetheme{metropolis}
\metroset{block=fill}
\usecolortheme{default}
\definecolor{darkmidnightblue}{rgb}{0.0, 0.2, 0.4}
\definecolor{LightGray}{gray}{0.9}


%------------------------------------------------------------
%This block of code defines the information to appear in the
%Title page
\title[Database Principles and Applications] %optional
{数据库原理与应用}

\subtitle{高级SQL(2)}

\author[CHEN Zhongpu] % (optional)
{CHEN Zhongpu}

\institute[] % (optional)
{
  School of Computing and Artificial Intelligence \\
  \href{mailto:zpchen@swufe.edu.cn}{zpchen@swufe.edu.cn}
}

\date[] % (optional)
{SWUFE, Fall 2022}

%End of title page configuration block
%------------------------------------------------------------


%------------------------------------------------------------
%The next block of commands puts the table of contents at the 
%beginning of each section and highlights the current section:

% \AtBeginSection[]
% {
%   \begin{frame}
%     \frametitle{Table of Contents}
%     \tableofcontents[currentsection]
%   \end{frame}
% }
%------------------------------------------------------------


\begin{document}

%The next statement creates the title page.
\frame{\titlepage}

%---------------------------------------------------------
%This block of code is for the table of contents after
%the title page
% \begin{frame}
% \frametitle{Table of Contents}
% \tableofcontents
% \end{frame}
%--------------------------------------------------------

\begin{frame}
    \frametitle{复习}
    \begin{itemize}
        \item 高级数据类型(时间和日期)
        \item 类型转化
        \item 授权
    \end{itemize}

\end{frame}
\begin{frame}
    \frametitle{小测试}
给定某商场2021年的销售关系\texttt{sale2021(id, product\_id, time, price, unit)}:
\begin{itemize}
    \item 求出该商场2021年的总销售额。
    \item 求出该商场销量前5的产品。
    \item 找到销量最高的月份。
\end{itemize}
\begin{table}
    \begin{tabular}{lllll}
      \toprule
      id & product\_id & time & price & unit \\
      \midrule
      1 & A1 & 2021-01-01 12:01:00 & 0.3 & 10 \\
      2 & B2 & 2021-01-01 13:00:00 & 2.7 & 1 \\
      \bottomrule
    \end{tabular}
\end{table}
\end{frame}

{
    % \usebackgroundtemplate{\transparent{0.3}{\begin{picture}
    %     \includegraphics[height=0.7\paperheight]{cover}
    % \end{picture}    
    % }}
\usebackgroundtemplate{
  \tikz[overlay,remember picture] 
  \node[opacity=0.3, at=(current page.south east),anchor=south east, yshift=2cm,xshift=4cm] {
    \includegraphics[height=0.6\paperheight]{cover}};
}
    \begin{frame}
        \section{\textcolor{darkmidnightblue}{1. 函数与过程}}
        Functions and procedures
    \end{frame}

}

\begin{frame}[fragile]
    \frametitle{Functions}

    \begin{columns}
        \column{.4\textwidth}
        \begin{minted}[bgcolor=LightGray, baselinestretch=1]{python}
def add(i, j):
    return i + j
    \end{minted}
        \column{.6\textwidth}
    \begin{minted}[bgcolor=LightGray, baselinestretch=1]{java}
public int add(int i, int j) {
    return i + j;
}
     \end{minted} 

    \end{columns}


    \begin{minted}[bgcolor=LightGray, baselinestretch=1]{sql}
SELECT AVG(salary)
FROM instructor
GROUP BY length(name);
    \end{minted}

\end{frame}

\begin{frame}[fragile]
    \frametitle{1.1 定义函数}

    \begin{minted}[bgcolor=LightGray, baselinestretch=1]{sql}
SELECT add(1, 2);

CREATE FUNCTION
add(integer, integer) RETURNS integer
AS 'select $1 + $2;'
LANGUAGE SQL;
    \end{minted}

可以在DataGrip左侧的的\texttt{「public」}$\rightarrow$\alert{\texttt{「routines」}}看到自定义的函数。

\faIcon{lightbulb}思考:如何删除自定义的函数?
\end{frame}


\begin{frame}[fragile]
也可以使用\alert{\texttt{\$\$}},而不使用单引号,并且也可以给参数取名,甚至写成类似函数的形式:

\begin{minted}[bgcolor=LightGray, baselinestretch=1]{sql}
CREATE FUNCTION multiply(i integer, j integer)
RETURNS integer AS
$$
BEGIN
    RETURN i * j;
END;
$$
LANGUAGE plpgsql;
\end{minted}

\end{frame}

\begin{frame}[fragile]
    \frametitle{例子}
找到教师数大于12的所有学院的名称和预算。

\pause

假设有一个函数\alert{\texttt{deptCount()}},可以返回给定学院的教师数:

\begin{minted}[bgcolor=LightGray, baselinestretch=1]{sql}
SELECT dept_name, budget
FROM department
WHERE deptCount(dept_name) > 12;
\end{minted}
\end{frame}

\begin{frame}[fragile]
    \begin{minted}[bgcolor=LightGray, baselinestretch=1]{sql}
CREATE FUNCTION deptCount(varchar(20))
RETURNS integer as
    $$
    DECLARE d_count integer;
        BEGIN
        SELECT count(*) INTO d_count
        FROM instructor
        WHERE instructor.dept_name = $1;
        RETURN d_count;
        END;
    $$ LANGUAGE plpgsql;
    \end{minted}

\faIcon{search} 查阅资料,理解\texttt{SELECT INTO}的语法。
\end{frame}

\begin{frame}[fragile]
    \frametitle{1.2 定义过程}

    \begin{minted}[bgcolor=LightGray, baselinestretch=1]{sql}
CREATE PROCEDURE add2(integer, integer)
AS
$$
SELECT $1 + $2;
$$ LANGUAGE SQL;
    \end{minted}

\faIcon{code} 定义上面的过程后,试着运行\texttt{SELECT add2(1, 2)},你有什么发现?
\end{frame}

\begin{frame}[fragile]
    \frametitle{使用「过程」的场景}
现实世界中一个动作往往涉及到数据库的多次读写。比如,在淘宝下单时,数据库需要1)添加订单;2)添加配送信息。
    
\begin{minted}[bgcolor=LightGray, baselinestretch=1]{sql}
CREATE PROCEDURE 
    buy(id varchar(50), product text, address text)
AS $$
    INSERT INTO order(id, product) VALUES($1, $2);
    INSERT INTO deliver VALUES($2, $3);
    $$ 
LANGUAGE SQL;
\end{minted}

\end{frame}

\begin{frame}
    \frametitle{练习}
西财的信息中心拟开发一个数据库过程\texttt{welcomeToCS()},使得在录入计算机系的新生基本信息的同时,自动完成当年秋季学期CS-101课程(课程段为1)的选课。

请实现上述需求。(参数包括学号和姓名)

\end{frame}

{\setbeamercolor{palette primary}{fg=black, bg=yellow}
\begin{frame}[standout]
    函数(function)返回结果,而过程(procedure)主要用来执行一些操作。

    {\small \faIcon{fire} 注意:各个DBMS的具体语法与标准SQL差异较大。}
\end{frame}
}

\begin{frame}[fragile]
    \section{\textcolor{darkmidnightblue}{2. 触发器 (trigger)}}
    A \alert{trigger} is a statement that the system executes automatically as a side effect of a modification to the database.
\end{frame}

\begin{frame}
    \frametitle{触发器}
To define a trigger, we must:

\begin{itemize}
    \item Specify when a trigger is to be executed. This is broken up into an \textbf{event} that causes the trigger to be checked and a \textbf{condition} that must be satisfied for trigger execution to proceed.
    \item Specify the \textbf{actions} to be taken when the trigger executes.
\end{itemize}

\faIcon{bell} 比如,当学生某门课及格后(\texttt{takes}),需要自动增加她所修的总学分(\texttt{student})。
\end{frame}

\begin{frame}[fragile]
    \section{\textcolor{darkmidnightblue}{3. 使用程序语言访问数据库}}
\begin{center}
    {\Huge \faIcon{java} \faIcon{python} \faIcon{rust} \faIcon{php}}

{\Huge \faIcon{arrows-alt-v}}

{\Huge \faIcon{database}}
\end{center}

\end{frame}

\begin{frame}
    \frametitle{3.1 背景知识}
\begin{exampleblock}{B-S构架}
    DBMS是经典的B-S构架。我们使用的SQL Shell或是DataGrip、Navicat等软件仅仅是个客户端。
\end{exampleblock}

类似的,使用编程语言(如Python、Java等)编写的程序也可以作为客户端与DBMS进行交互。

\faIcon{eye} \textbf{小任务}:在DBGrip的左侧,在\texttt{mydb@localhost}上右键,点击「Properties」,你看到是URL是什么?你觉得需要什么信息才能唯一地标识一个数据库?

\end{frame}

\begin{frame}

    https://github.com/ChenZhongPu/

    \begin{tikzpicture}[every node/.style={scale=1.3}]
        \node[red](proxy){jdbc:postgresql};
        \node(t1)[right=of proxy,xshift=-1cm]{://};
        \node[blue, right=of t1, xshift=-1cm](server){localhost};
        \node(t2)[right=of server,xshift=-1cm]{:};
        \node(port)[olive,right=of t2, xshift=-1cm]{5432};
        \node(t3)[right=of port, xshift=-1cm]{:};
        \node(db)[right=of t3, xshift=-1cm, violet]{mydb};

        \node(proxyt)[above=of proxy]{协议};
        \draw[->, thick](proxyt) -- (proxy);

        \node(servert)[above=of server]{服务器};
        \draw[->, thick](servert) -- (server);
        
        \node(portt)[above=of port]{端口};
        \draw[->, thick](portt) -- (port);
    \end{tikzpicture}

当然,访问数据库的时候还需要用户名和密码。
\end{frame}

\begin{frame}
    \frametitle{3.2 连接SQL的标准}
\begin{itemize}
    \item JDBC
    \item ODBC
\end{itemize}
实现上述标准的程序一般称为数据库驱动(driver或connector),它们的作用是将程序中的SQL发送给DBMS,再将DBMS的执行结果返回给程序。

\begin{tikzpicture}
    \node[fill=yellow,blur shadow={shadow xshift=-0.5ex},
    text width=16em,anchor=south west,rounded corners]
    {不同DBMS的驱动是不同的。};
\end{tikzpicture} 

\end{frame}

\begin{frame}[fragile]
    \frametitle{3.3 Python访问数据库}
    Python中访问PostgreSQL常用的驱动是\alert{\texttt{psycopg2}}。
    
\begin{verbatim}
pip3 install psycopg2
python3 main.py 
\end{verbatim}

\faIcon{code}\textbf{小任务}:阅读\href{https://github.com/ChenZhongPu/db-swufe/blob/master/10_advanced/python-demo/main.py}{main.py},并试着完成\texttt{delete()}功能。
\end{frame}

\begin{frame}[fragile]
    \frametitle{3.4 ORM}
但是,直接在程序里面嵌入SQL代码对于大型项目是个噩梦。

\begin{itemize}
    \item SQL代码无法在编译时验证是否正确
    \item 无法屏蔽不同DBMS的差异
\end{itemize}
    


\begin{minted}[bgcolor=LightGray, baselinestretch=1]{python}
cur.execute("SELECT * FROM department")
# Retrieve query results
records = cur.fetchall()
\end{minted}

为了解决这个问题,ORM(object-relational mapping)被提出。
\end{frame}

\begin{frame}[fragile]
Python中最流行的ORM框架是\href{https://www.sqlalchemy.org/}{SQLAlchemy}。


\begin{minted}[bgcolor=LightGray, baselinestretch=1]{python}
dept = session.query(Department)
              .filter_by(building='Watson')
              .first()

depts = session.query(Department)
               .filter(Department.dept_name.like('%s'))
\end{minted}

课下有兴趣可以阅读\href{https://github.com/ChenZhongPu/db-swufe/blob/master/10_advanced/python-orm/main-orm.py}{main-orm.py}。
\end{frame}

\begin{frame}[fragile]
    \section{\textcolor{darkmidnightblue}{4. SQL注入攻击(SQL Injection)}}
    (恶意的)SQL语句插入了特殊的字符串,服务器执行后造成了意想不到的效果。
\end{frame}

\begin{frame}[fragile]
    \frametitle{回顾}
\begin{exampleblock}{Homework 1}
假设数据库中存储用户名和密码的关系模式是\texttt{users(name, pswd, gender)},请结合关系代数简述实现用户登录的思路。
\end{exampleblock}

\begin{minted}[bgcolor=LightGray, baselinestretch=1]{python}
sql = f"SELECT * FROM users WHERE name = '{name}'
        AND pswd = '{pswd}'"
\end{minted}

\begin{minted}[bgcolor=LightGray, baselinestretch=1]{sql}
SELECT *
FROM users
WHERE name='zhongpu' AND pswd='123abc';
\end{minted}

\end{frame}

\begin{frame}[fragile]
    \frametitle{登录SQL}

    \begin{minted}[bgcolor=LightGray, baselinestretch=1]{python}
name = "zhongpu"
pswd = "' or '1'='1"
sql = f"SELECT * FROM users WHERE name='{name}'
        AND pswd='{pswd}'"
    \end{minted}  

\faIcon{code} 上面的字符串拼接的结果是什么?并试着运行改SQL语句。
\end{frame}

\begin{frame}[fragile]

如果密码是\alert{\texttt{' or '1'='1}},SQL语句为:

\begin{minted}[bgcolor=LightGray, baselinestretch=1]{sql}
SELECT * 
FROM users 
WHERE name='zhongpu' 
AND pswd='' or '1'='1';
\end{minted}

\pause
如果密码是\alert{\texttt{' or 1='1}},SQL语句为:

\begin{minted}[bgcolor=LightGray, baselinestretch=1]{sql}
SELECT *
FROM users
WHERE name='zhongpu'
AND pswd='' or 1='1';
\end{minted}

\end{frame}

{\setbeamercolor{palette primary}{fg=black, bg=yellow}
\begin{frame}[standout]
使用简单的字符串拼接构造的SQL语句有SQL注入的风险!

可以使用ORM或PreparedStatement避免这个问题。
\end{frame}
}


\begin{frame}[fragile]
    \section{\textcolor{darkmidnightblue}{总结}}

\begin{itemize}
    \item 函数与过程
    \item 动态SQL
    \item SQL注入
\end{itemize}
\end{frame}

\end{document}