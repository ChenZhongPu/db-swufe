\documentclass[aspectratio=169, 14pt]{beamer}
\usepackage[utf8]{inputenc}
\usepackage{xeCJK}
\usepackage{tipa}
\usepackage{graphicx}
\usepackage{transparent}
\usepackage[ruled, lined, linesnumbered, commentsnumbered]{algorithm2e}
\usepackage{tikz}
\usetikzlibrary{calc,shadows.blur}
\usetikzlibrary{matrix,backgrounds}
\usetikzlibrary{arrows, positioning}
\usepackage{minted}
\usepackage{fontawesome5}
\usepackage{booktabs}
\usepackage{hyperref}
\hypersetup{
    colorlinks=true,
    linkcolor=blue,
    filecolor=magenta,      
    urlcolor=cyan,
    }
\urlstyle{same}
\usetheme{metropolis}
\metroset{block=fill}
\usecolortheme{default}
\definecolor{darkmidnightblue}{rgb}{0.0, 0.2, 0.4}
\definecolor{LightGray}{gray}{0.9}
\definecolor{usecolor}{RGB}{141,143,168}
\newcommand\grid[1]{%
\fontsize{80}{80}
\begin{tikzpicture}[baseline=(char.base)]
  \path[use as bounding box]
    (0,0) rectangle (1em,1em);
  \draw[help lines,step=0.5em]
    (0,0) grid (1em,1em);
  \draw[help lines,dashed]
    (0,0) -- (1em,1em)  (0,1em) -- (1em,0);
  \node[inner sep=0pt,anchor=base west]
    (char) at (0em,\gridraiseamount) {#1};
\end{tikzpicture}}                                                                                                                       
% \gridraiseamount is a font-specific value
\newcommand\gridraiseamount{0.12em}

%------------------------------------------------------------
%This block of code defines the information to appear in the
%Title page
\title[Database Principles and Applications] %optional
{数据库原理与应用}

\subtitle{课程介绍}

\author[CHEN Zhongpu] % (optional)
{CHEN Zhongpu}

\institute[] % (optional)
{
  School of Computing and Artificial Intelligence \\
  \href{mailto:zpchen@swufe.edu.cn}{zpchen@swufe.edu.cn}
}

\date[] % (optional)
{SWUFE, Spring \the\year{}}

%End of title page configuration block
%------------------------------------------------------------


%------------------------------------------------------------
%The next block of commands puts the table of contents at the 
%beginning of each section and highlights the current section:

% \AtBeginSection[]
% {
%   \begin{frame}
%     \frametitle{Table of Contents}
%     \tableofcontents[currentsection]
%   \end{frame}
% }
%------------------------------------------------------------


\begin{document}

%The next statement creates the title page.
\frame{\titlepage}

%---------------------------------------------------------
%This block of code is for the table of contents after
%the title page
% \begin{frame}
% \frametitle{Table of Contents}
% \tableofcontents
% \end{frame}
%--------------------------------------------------------
\begin{frame}
	\frametitle{关于我:陈中普}

	\begin{block}{\faIcon{book} 研究方向}
		数据库,大语言模型
	\end{block}

	\begin{block}{\faIcon{building} 办公室}
		格致楼J-310
	\end{block}

	\begin{block}{\faIcon{home} 主页}
		\href{https://zhongpu.info}{https://zhongpu.info}
	\end{block}

\end{frame}

{
% \usebackgroundtemplate{\transparent{0.3}{\begin{picture}
%     \includegraphics[height=0.7\paperheight]{cover}
% \end{picture}    
% }}
\usebackgroundtemplate{
	\tikz[overlay,remember picture]
	\node[opacity=0.3, at=(current page.south east),anchor=south east, yshift=2cm,xshift=2cm] {
		\includegraphics[height=0.6\paperheight]{cover}};
}
\begin{frame}
	\section{\textcolor{darkmidnightblue}{关于课程}}
\end{frame}
}

\begin{frame}
	\frametitle{关于课程}
	\begin{block}{数据库}
		\alert{数据库}是计算机、软件工程、数据科学等专业的核心课程。

		它向上能支撑互联网、银行等不同行业的应用,向下能触及存储、查询优化、编译、分布式计算等计算机核心技术。
	\end{block}
	\grid{顶} \grid{天} \grid{立} \grid{地}
\end{frame}


\begin{frame}
	\frametitle{预备知识}
	原则上几乎不需要预备知识,掌握\alert{阅读文档}(尤其是英文文档)和\alert{安装软件}的能力即可。
	\pause
	\begin{block}{重点}
		本课程是应用导向的,因此存储、查询优化、事务等技术几乎不涉及。重点是:
		\begin{itemize}
			\item  如何使用数据库
			\item  如何设计数据库
		\end{itemize}

	\end{block}
\end{frame}

\begin{frame}[fragile]
	\begin{center}
		\includegraphics[height=.99\paperheight]{image/system}
	\end{center}
\end{frame}

\begin{frame}
	\frametitle{参考书籍}
	\href{https://book.douban.com/subject/30345517/}{Database System Concepts}, 7th edition, by Abraham Silberschatz, Henry F Korth, and S Sudarshan.
	\begin{center}
		\includegraphics[height=.5\paperheight]{image/dsc-book}
	\end{center}
\end{frame}

\begin{frame}
	\frametitle{非参考书籍}
	\href{https://book.douban.com/subject/26317662/}{数据库系统概论},第五版,by 王珊,萨师煊。
	\begin{center}
		\includegraphics[height=.5\paperheight]{image/book2}
	\end{center}

\end{frame}

\begin{frame}
	\frametitle{评价标准}

	\begin{tikzpicture}
		\colorlet{good}{green!75!black}
		\colorlet{bad}{red}
		\colorlet{neutral}{black!60}
		\colorlet{none}{blue!75!black}

		\node[align=center,text width=3cm]{考核};

		\begin{scope}[line width=4mm,rotate=270]
			\draw[good]          (216:2cm) arc (216:360:2cm);
			\draw[bad]  (0:2cm)  arc (0:108:2cm);
			\draw[none] (108:2cm) arc (108:216:2cm);

			\newcount\mycount
			\foreach \angle in {0,72,...,3599}
				{
					\mycount=\angle\relax
					\divide\mycount by 10\relax
					\draw[black!15,thick] (\the\mycount:18mm) -- (\the\mycount:22mm);
				}

			\draw (-90:2.2cm) node[left] {期末Project (40\%)};
			\draw (65:2.2cm) node[right] {作业 (30\%)};
			\draw (130:2.2cm) node[right] {期末测试 (30\%)};
		\end{scope}
	\end{tikzpicture}
	\begin{tikzpicture}
		\node[fill=red!80, text=white, blur shadow={shadow xshift=-0.5ex},
			text width=8em,anchor=south west,rounded corners, ]
		{严禁抄袭!
		};
	\end{tikzpicture}
	其中,Project成绩暂定是学生自行评定。
\end{frame}
\begin{frame}
	\frametitle{软件工具}
	PostgreSQL 17: \emph{The World's Most Advanced Open Source Relational Database}

	\includegraphics[width=.9\paperwidth]{image/pg-home}
\end{frame}

\begin{frame}
	\href{https://db-engines.com/en/ranking}{DB-Engines Ranking}

	\begin{table}
		\caption{DB Rank}
		\begin{tabular}{lll}
			\toprule
			Rank & DBMS                         & Database Model \\
			\midrule
			1    & \textcolor{blue}{Oracle}     & Relational     \\
			2    & \textcolor{blue}{MySQL}      & Relational     \\
			3    & \textcolor{blue}{SQL Server} & Relational     \\
			4    & \textcolor{blue}{PostgreSQL} & Relational     \\
			5    & \textcolor{blue}{MongoDB}    & Document       \\
			6    & \textcolor{blue}{Redis}      & Key value      \\
			\bottomrule
		\end{tabular}
	\end{table}
\end{frame}

\begin{frame}
	\includegraphics[width=.9\paperwidth]{image/pg-mysql1}
	\begin{itemize}
		\item 会什么选什么。PG 功能强大,但是我只会玩 MySQL 所以我用 MySQL 。
		\item 选 mysql,用 pg 的人一般不会有这个疑问。
	\end{itemize}
\end{frame}

\begin{frame}
	\frametitle{软件工具}
	\href{https://www.jetbrains.com/datagrip/}{DataGrip},Navicat,pgAdmin,...

	\begin{center}
		\includegraphics[width=.55\paperwidth]{image/datagrip}
	\end{center}

\end{frame}
\begin{frame}
	\frametitle{AI时代下的基础数据库教学}

	\begin{columns}
		\column{0.6\textwidth}
		\begin{center}
			\includegraphics[width=.5\paperwidth]{image/text2sql}
		\end{center}
		\column{0.4\textwidth}
		ChatGPT、DeekSeek和Claude等大模型的Text2SQL能力足够应付非高阶任务。
	\end{columns}


	{\tiny Li, Boyan, et al. "The Dawn of Natural Language to SQL: Are We Fully Ready?." Proceedings of the VLDB Endowment 17.11 (2024): 3318-3331.}
\end{frame}

\begin{frame}
	\begin{center}
		\includegraphics[width=.7\paperwidth]{image/chatgpt}
	\end{center}

	{\tiny \href{https://chatgpt.com/share/67bef99e-70e8-8010-abe7-6732a19093c4}{ChatGPT}}

\end{frame}
\end{document}
